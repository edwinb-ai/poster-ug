%%%%%%%%%%%%%%%%%%%%%%%%%%%%%%%%%%%%%%%%
% Class options                        %
%%%%%%%%%%%%%%%%%%%%%%%%%%%%%%%%%%%%%%%%
% Orientation:                         %
% portrait (default), landscape        %
%                                      %
% Paper size:                          %
% a0paper (default), a1paper, a2paper, %
% a3paper, a4paper, a5paper, a6paper   %
%                                      %
% Language:                            %
% english (default), norsk             %
%%%%%%%%%%%%%%%%%%%%%%%%%%%%%%%%%%%%%%%%
\documentclass{uioposter}


\usepackage{lipsum}                                % Dummy text
\usepackage[figwidth = 0.98\linewidth]{todonotes}  % Dummy image (and more!)
\usepackage[absolute, overlay]{textpos}            % Figure placement
\usepackage{hyperref}
\setlength{\TPHorizModule}{\paperwidth}
\setlength{\TPVertModule}{\paperheight}


\title{Applications of Machine Learning to Soft Matter}
\author
{%
    Edwin Bedolla\inst{1, $\dagger$}
    \and
    Edmundo Vázquez\inst{1}
    \and
    Ramón Castañeda-Priego\inst{1,2}
}
%% Optional:
\institute
{
    \inst{1} División de Ciencias e Ingenierías
    \and
    \inst{2} Departamento de Ingeniería Física
}
% Or:
%\institute{Contact information}


%% Remove footline:
\setbeamertemplate{footline}{}


\begin{document}
\begin{frame}
\begin{columns}[onlytextwidth]


\begin{column}{0.5\textwidth - 1.5cm}
    \begin{alertblock}{Abstract}
        With recent developments in Machine Learning (ML) technology, Soft Matter (SM) has been adopting several methodologies to enhance and provide new insights into difficult problems within the field. In this work, we showcase two recent applications of ML to SM systems, namely a data-driven ML model to show how structure is important to glassy dynamics; and a Deep Learning techinique to train ML models in order to detect topological defects in liquid crystals with data from video microscopy experiments. We briefly discuss the advantages and shortcomings of these frameworks, with the hope that future research can adopt some of these ideas and techniques in order to empower problem solving within SM.
    \end{alertblock}

    \begin{block}{Introduction}
        Within the Condensed Matter Physics field, Machine Learning (ML) is currently used in several research areas, ranging from
        quantum matter~\cite{carrasquilla_2020}, solid state physics~\cite{schmidt2019recent}, and more importantly, Soft Matter (SM). 
        The need to use ML is directly related to the amount of data that usual techniques
        create, in the form of experiments and computer simulations.
        Furthermore, there is a motivation that by using ML models, coupled with physics-designed
        features, new insights and physical parameters can be found from the datasets.
        For a more complete overview of the topic, we
        refer the reader to a recent review~\cite{bedolla2020}.

        Most of the applications that have been used deal almost exclusively with
        Hard Matter systems, such as spin systems like the $XY$ model, the Potts model,
        the Ising model, and others. This fact is due to ML models being developed
        mainly for grid-like data, such as images in computer vision. 
        Due to the fact that spin systems are grid-like data, ML models can be
        readily applied to them, although care must be taken when training and other
        technical details to ensure that the results obtained are physicaly meaningful
        and correct.

        On the other hand, SM has seen very little applications due
        to the fact that modern ML technology does not handle off-lattice and
        continuous data easily. Nevertheless, some work has been carried out
        in developing carefuly crafted features from simulation and experimental
        data that have boosted ML applications in the field of SM.

        In this short review, we will discuss two such applications, one that deals
        with glassy dynamics and another that deals with liquid crystal structure
        from experimental data.
    \end{block}

    \begin{block}{Results}
        % TODO: Agregar una figura de SVMs
    \end{block}
\end{column}


\begin{column}{0.5\textwidth - 1.5cm}
    \begin{block}{Conclusions}
        \lipsum[4]
    \end{block}

    \begin{block}{Acknowledgements}
        This work was financially supported by Conacyt grant 287067.
    \end{block}

    \begin{block}{References}
        \begin{thebibliography}{99}
            \bibitem{carrasquilla_2020}
            Carrasquilla, J.,
            Machine learning for quantum matter,
            \emph{Advances in Physics: X},
            5, pp. 1797528, 2020.
            
            \bibitem{schmidt2019recent}
            Schmidt, J., and Marques, M. and Botti, S. and Marques, M. AL.,
            Recent advances and applications of machine learning in solid-state materials science,
            \emph{npj Computational Materials},
            5, pp. 1-55, 2019.
            
            \bibitem{bedolla2020}
            Bedolla, E., and Padierna, L.C., and Casta{\~n}eda-Priego, R.,
            Machine Learning for Condensed Matter Physics,
            \emph{Journal of Physics: Condensed Matter},
            2020.
            
            \end{thebibliography}
    \end{block}

    \begin{block}{Contact information}
        $\dagger$ Corresponding Author: \href{mailto:ea.bedollamontiel@ugto.mx}{ea.bedollamontiel@ugto.mx}
        \\
    \end{block}
\end{column}


\end{columns}


\end{frame}
\end{document}